\documentclass[a4paper,anonymous,UKenglish,cleveref,autoref,thm-restate]{lipics-v2021}

% if images are in a different directory



% the mandatory bibstyle
\bibliographystyle{plainurl}

% define a \citenum command
% that only prints the number associated
% to the citation **without brackets**
% so \cite{foo} will print [1]
% and \citenum{foo} will print 1
% \newcommand{\citenum}[1]{\cite{#1}}

% START OF CUSTOM MACROS

\newcommand{\tightlist}{%
    \setlength{\itemsep}{0pt}\setlength{\parskip}{0pt}\setlength{\parsep}{0pt} %
}
% Create a proof environment for results in appendix,
% that take a label of a theorem in the main text, and
% produce a proof
% \begin{proof}[Proof of \cref{the-label}]
% \phantomsection\label{the-label:proof}
% content
% \end{proof}
\NewDocumentEnvironment{proofof}{o}{%
    \IfValueTF{#1}{%
        \begin{proof}[Proof of \cref{#1} on page \pageref{#1}]
        \phantomsection\label{#1:proof}
    }{%
        \begin{proof}
    }
    \let\oldqedsymbol\qedsymbol
    \renewcommand\qedsymbol{\hyperref[#1]{\oldqedsymbol}}
}{%
    \end{proof}
    \renewcommand\qedsymbol{\oldqedsymbol}
}

% Refers to the proof of a result
\NewDocumentCommand{\proofref}{m}{%
    % if appendices are present, then refer to the appendix
    % otherwise do nothing.
    % \hyperref[#1:proof]{Proof of \cref{#1}}
    \IfRefUndefinedExpandable{#1:proof}{}{%
        \hfill\hyperref[#1:proof]{(Proof p.\pageref{#1:proof})}
    }
}

% END OF CUSTOM MACROS

\title{Sample Article for LIPIcs}
\titlerunning{Sample LIPIcs Running}

\author{John Q. Public}{Dummy University Computing
Laboratory}{dummy-email}{dummy-orcid}{}
\author{John Q. Public}{Dummy University Computing
Laboratory}{dummy-email}{dummy-orcid}{some funding}

\authorrunning{John Q. Public, John Q. Public}

\Copyright{John Q. Public, John Q. Public}


\ccsdesc[100]{Theory of computation}
\ccsdesc[100]{Automata theory}

\keywords{First, Second, Third}

\category{Invited Paper} %optional, e.g. invited paper

\relatedversiondetails[
                       ]{Preprint}{https://arxiv.org/abs/XXX} %linktext and cite are optional

\supplementdetails[linktext={},
                   cite=,
                   subcategory={},
                   swhid={}]{Data}{http://dx.doi.org/10.4230/LIPIcs.xxx.xxx.xxx}



\nolinenumbers

\hideLIPIcs

% we set the mode to paper if lipics.review-mode or lipics.final-mode are set
% if lipics.arxiv-mode is set, we set the mode to electronic
% otherwise we use the composition mode
%\usepackage[capitalise,noabbrev,nameinlink]{cleveref}
\usepackage[cleveref,xcolor,hyperref,electronic]{knowledge}
\knowledgeconfigure{notion}

% TODO: take knowledge options into account
\knowledge{notion}
 | this
 | that
 | those@testing-scope

%Editor-only macros:: begin (do not touch as author)%%%%%%%%%%%%%%%%%%%%%%%%%%%%%%%%%%
\EventEditors{John Q. Open and Joan R. Access}
\EventNoEds{2}
\EventLongTitle{42nd Conference on Very Important Topics (CVIT 2016)}
\EventShortTitle{CVIT 2016}
\EventAcronym{CVIT}
\EventYear{2016}
\EventDate{December 24--27, 2016}
\EventLocation{Little Whinging, United Kingdom}
\EventLogo{}
\SeriesVolume{42}
\ArticleNo{23}
%%%%%%%%%%%%%%%%%%%%%%%%%%%%%%%%%%%%%%%%%%%%%%%%%%%%%%

\begin{document}

\maketitle

\begin{abstract}
    Abstract of the paper
\end{abstract}


\section{Introduction}\label{introduction}

\subsection{Theorems and proofs}\label{theorems-and-proofs}

\subsubsection{Optional Theorem title}\label{optional-theorem-title}

Theorem statement

\section{Proof}\label{proof}

proof statement

\phantomsection\label{firstclaimlabel}
Let \(x\) be a variable, we can do this and this and that.

Then in particular:

\begin{enumerate}
\def\labelenumi{\roman{enumi}.}
\tightlist
\item
  \(x\) is a variable.
\item
  \(x\) is a variable.
\end{enumerate}

All of the above are equivalent to \(x\) being a variable.

\section{Proof Sketch}\label{proof-sketch}

\subsection{Knowledges}\label{knowledges}

We can introduce knowledges with \intro{this} and later on refer to
those using \kl{this}. If for some strange reason we want to introduce
them twice, we can use \reintro{that}.

If we want to use a scoped knowledge, we can like this
\kl(testing-scope){those}.

\subsection{Citations}\label{citations}

We start by citing a paper {[}\cite{DBLP:journals/cacm/Knuth74}{]}.

We can also cite them like this
\nocite{DBLP:journals/cacm/Knuth74}\hyperlink{cite.DBLP:journals/cacm/Knuth74}{Knuth}.
Which becomes a tiny bit more impressive using a lot of names such as
\nocite{DBLP:conf/focs/HopcroftPV75}\hyperlink{cite.DBLP:conf/focs/HopcroftPV75}{Hopcroft,
Paul, and Valiant}.

Note that we have the full power of the pandoc citation syntax. In
particular we can
{[}see \cite{DBLP:journals/cacm/Knuth74} because \cite{DBLP:conf/focs/HopcroftPV75} 
has Theorem 1.6{]}.

\section{Main part}\label{main-part}

Imagine some text followed by a theorem

{Theorem 1} (good omens).

There are good omens.

Good omen 1.

Good omen 2.

Proof

Proof of Theorem 1.

\section{Introduction}\label{introduction-1}

Hello.

\section{Preliminaries}\label{preliminaries}

% add the bibliographies
\bibliography{lipics-v2021-sample-article.bib}



\end{document}
